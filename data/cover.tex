\thusetup{
  %******************************
  % 注意:
  %   1. 配置里面不要出现空行
  %   2. 不需要的配置信息可以删除
  %******************************
  %
  %=====
  % 秘级
  %=====
  secretlevel={秘密},
  secretyear={10},
  %
  %=========
  % 中文信息
  %=========
  ctitle={基于多步决策的文本情感识别},
  cdegree={工学硕士},
  cdepartment={计算机科学与技术系},
  cmajor={计算机科学与技术},
  cauthor={梁锡豪},
  csupervisor={徐明星副教授},
  % cassosupervisor={陈文光教授}, % 副指导老师
  % ccosupervisor={某某某教授}, % 联合指导老师
  % 日期自动使用当前时间,若需指定按如下方式修改:
  % cdate={超新星纪元},
  %
  % 博士后专有部分
  % cfirstdiscipline={计算机科学与技术},
  % cseconddiscipline={系统结构},
  % postdoctordate={2009年7月——2011年7月},
  % id={编号}, % 可以留空: id={},
  % udc={UDC}, % 可以留空
  % catalognumber={分类号}, % 可以留空
  %
  %=========
  % 英文信息
  %=========
  etitle={Multi-step-decision-making Based Text Sentiment Analysis},
  % 这块比较复杂,需要分情况讨论:
  % 1. 学术型硕士
  %    edegree:必须为Master of Arts或Master of Science(注意大小写)
  %             “哲学、文学、历史学、法学、教育学、艺术学门类,公共管理学科
  %              填写Master of Arts,其它填写Master of Science”
  %    emajor:“获得一级学科授权的学科填写一级学科名称,其它填写二级学科名称”
  % 2. 专业型硕士
  %    edegree:“填写专业学位英文名称全称”
  %    emajor:“工程硕士填写工程领域,其它专业学位不填写此项”
  % 3. 学术型博士
  %    edegree:Doctor of Philosophy(注意大小写)
  %    emajor:“获得一级学科授权的学科填写一级学科名称,其它填写二级学科名称”
  % 4. 专业型博士
  %    edegree:“填写专业学位英文名称全称”
  %    emajor:不填写此项
  edegree={Doctor of Engineering},
  emajor={Computer Science and Technology},
  eauthor={Liang Xihao},
  esupervisor={Professor Xu Mingxing},
  % eassosupervisor={Chen Wenguang},
  % 日期自动生成,若需指定按如下方式修改:
  % edate={December, 2005}
  %
  % 关键词用“英文逗号”分割
  ckeywords={情感分析, 反讽识别, 深度学习,集成学习},
  ekeywords={Sentiment analysis, irony detection, deep learning, ensemble learning}
}

% 定义中英文摘要和关键字
\begin{cabstract}

自Web2.0普及后,人们逐渐习惯在互联网的各种平台上分享他们的想法和情感。透过对这些媒体进行情感分析,我们可以得知人们对特定人事物的想法和态度。对人们的想法快速作出响应能够带来相应的商业价值和政治价值,其相关技术也因此得到了重视。而文本作为社交平台上的主要媒体之一,面向文本的情感识别在近年也成为了热门的研究领域。本论文探讨了文本情感识别中的两个问题,主要内容如下:

\begin{enumerate}

\item {\bf 基于多步决策的多分类系统框架}。随着现实中应用场景变得复杂,需要解决的多分类问题越来越多。随着需要区分的类别越多,对数据拟合的难度更高,另外对识别性能的要求也变得复杂,譬如确保个别类别的召回率和正确率等。为此,我们提出了一种基于多步决策的多分类系统设计方法,把一个多分类问题拆解成多个子分类问题的叠加,再透过逐步回答每个子问题得出最终的识别结果。由于识別过程中每一步只关注一个子分类问题,所以我們能从局部调整系统的识别能力,並针对特殊的评价指标进行优化。另一方面,每一步的子問题可以分別采用不同的算法,以此结合多种模型的模型能力。


\item {\bf 用于结合上下文的多通道模型}。在文本情感识别中,有时候仅凭一段文本无法准确了解发言者想表达的意思和态度,在一些场景下我们会考虑引入文本的上下文作为提示。为此我们提出了一种多通道模型框架,让对识别目标起不同作用的上下文先分别经过不同的编码器进行编码提取出有用的特征,再考虑合并正文和上下文的信息得出识别结果。

\end{enumerate}

为了验证基于多步决策多分类系统框架,我们将其应用于面向微博的反讽识别,根据公开比赛SemEval-2018的任务三进行实验,结果显示我们的模型超过了当时排名第一的系统。
另外为了验证我们提出的多通道分类模型,我们将其应用于面向三轮对话的情感识别,根据公开比赛SemEval-2019的任务三进行实验,结果显示我们的模型达到了当时排名前十的性能。

% 关键词是为了文献标引工作、用以表示全文主要内容信息的单词或术语。关键词不超过 5
% 个,每个关键词中间用分号分隔。(模板作者注:关键词分隔符不用考虑,模板会自动处
% 理。英文关键词同理。)

\end{cabstract}

% 如果习惯关键字跟在摘要文字后面,可以用直接命令来设置,如下:
% \ckeywords{\TeX, \LaTeX, CJK, 模板, 论文}

\begin{eabstract}

Since the poularization of Web 2.0, people get used to share they thoughts and emotions on different online platforms. By analyzing the sentiment of these data, we can have an in sight into the people's opinion towards certain things. Vast commercial and political values can be obtained by fast response to the people's opinion, therefore related technologies are getting more attention. As text is one of the mostly used media on social platform, text sentiment analysis has become a popular research field. The thesis explores two problems in text sentiment analysis and it is organized as follows:

\begin{enumerate}

\item {\bf Multi-step-decision-making based multi-classification system framework}. As application scenes in the real world are becoming more complicated, more multi-classification problems are encountered. As the data are divided into more categories, it becomes more difficult for machine learning algorithms to fit the data. On the other hand, the requirement of predictive performance are getting more complicated, such as ensuring the recall rate and the precision level of certain categories. Hence, we propose a multi-step-decision-making based multi-classification system framework, which disassembles a multi-classification problem into a sequence of several sub-classification problems and classifies an object by answering the sub-questions step by step. As the classification system only focuses on a sub-question at each step, its performance can be adjusted locally and we can optimize the overall performance for certain evaluation metrics. On the other hand, algorithms can be chosen for each sub-problem individually, hence the information captured by different kinds of models can be combined in the system. 

\item {\bf Multi-channel model for manipulating contextual information}. Sometimes we may not exactly understand what a person want to express or how he feels only by reading his written message. To deal with this problem, some of the text sentiment analysis researches take into account the contextual clues. Therefore, we propose a multi-channel model. By feeding contextual information that play different parts of role into different feature encoders, features of different parts of the context are encoded and combined with that of the main text to get the identification result.

\end{enumerate}

To verify the effectiveness of the proposed multi-step-decision-making based multi-classification system framework, we apply it to irony detection in English tweets. Experiments are launched based on SemEval-2018 Task 3. Results show that our system exceeds the performance of the no. 1 system. On the other hand, to verify the effectiveness of our multi-channel model, we apply it to contextual emotion detection in text. Experiments are launched based on SemEval-2019 Task 3. Results show that our system reaches the top 10 performance.

\end{eabstract}

% \ekeywords{\TeX, \LaTeX, CJK, template, thesis}
