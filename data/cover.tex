\thusetup{
  %******************************
  % 注意:
  %   1. 配置里面不要出现空行
  %   2. 不需要的配置信息可以删除
  %******************************
  %
  %=====
  % 秘级
  %=====
  secretlevel={秘密},
  secretyear={10},
  %
  %=========
  % 中文信息
  %=========
  ctitle={清华大学学位论文 \LaTeX\ 模板\\使用示例文档 v\version},
  cdegree={工学硕士},
  cdepartment={计算机科学与技术系},
  cmajor={计算机科学与技术},
  cauthor={梁锡豪},
  csupervisor={徐明星副教授},
  % cassosupervisor={陈文光教授}, % 副指导老师
  % ccosupervisor={某某某教授}, % 联合指导老师
  % 日期自动使用当前时间,若需指定按如下方式修改:
  % cdate={超新星纪元},
  %
  % 博士后专有部分
  % cfirstdiscipline={计算机科学与技术},
  % cseconddiscipline={系统结构},
  % postdoctordate={2009年7月——2011年7月},
  % id={编号}, % 可以留空: id={},
  % udc={UDC}, % 可以留空
  % catalognumber={分类号}, % 可以留空
  %
  %=========
  % 英文信息
  %=========
  etitle={An Introduction to \LaTeX{} Thesis Template of Tsinghua University v\version},
  % 这块比较复杂,需要分情况讨论:
  % 1. 学术型硕士
  %    edegree:必须为Master of Arts或Master of Science(注意大小写)
  %             “哲学、文学、历史学、法学、教育学、艺术学门类,公共管理学科
  %              填写Master of Arts,其它填写Master of Science”
  %    emajor:“获得一级学科授权的学科填写一级学科名称,其它填写二级学科名称”
  % 2. 专业型硕士
  %    edegree:“填写专业学位英文名称全称”
  %    emajor:“工程硕士填写工程领域,其它专业学位不填写此项”
  % 3. 学术型博士
  %    edegree:Doctor of Philosophy(注意大小写)
  %    emajor:“获得一级学科授权的学科填写一级学科名称,其它填写二级学科名称”
  % 4. 专业型博士
  %    edegree:“填写专业学位英文名称全称”
  %    emajor:不填写此项
  edegree={Doctor of Engineering},
  emajor={Computer Science and Technology},
  eauthor={Liang Xihao},
  esupervisor={Professor Xu Mingxing},
  % eassosupervisor={Chen Wenguang},
  % 日期自动生成,若需指定按如下方式修改:
  % edate={December, 2005}
  %
  % 关键词用“英文逗号”分割
  ckeywords={意图识別, 情感分析, 反讽识別, 深度学习,集成学习},
  ekeywords={Opinion mining, sentiment analysis, irony detection, deep learning, ensemble learning}
}

% 定义中英文摘要和关键字
\begin{cabstract}

自Web2.0普及后,人们习惯在互联网的各种平台上透过不同的媒体来分享和传递他们的想法和情感。透过对这些媒体进行情感分析,我们可以得知人们的想法和态度。譬如透过分析产品评论了解用户对新产品是否满意,又或者分析社交平台上的舆论了解网民对新政策是否同意,以此快速响应,其中蕴含着丰富的商业价值和政治价值,其相关技术的研究价值在近年也得到了重视。而文本作为较常见的媒体之一,加之部分微博等社交平台推出了用户生产数据的采集服务,提供了充裕的文本数据资源,面向文本的情感识别研究得以更快的速度发展。然而随着在互联网上的交互方式变得丰富,文本的数据结构也变得多种多样,如微博上的短文本、在线实时聊天的两人或多人对话、讨论区里的帖子回复。在处理不同数据结构时,该如何提取当中的信息并加以运用来预测目标结果,成为了重要的研究课题。

本文主要探讨了两个不同数据结构下的情感识别问题,并提出了一个通用的文本情感识别系统设计方案。论文的主要内容如下:

\begin{itemize}

\item 面向微博的反讽识别。反讽作为一种特殊的修辞手法,起着把文本字面意思反转的效果。识别文本中反讽修辞的使用对于正确理解发言者的意图起着关键性的作用。本文研究了在微博中识别反讽的出现以及其对应类型,深入了解实验数据当中各个类型的样本的特性,分析单个分类器对各个类的识别能力,给出一个基于集成学习的反讽识别框架,以此在多个独立分类器的基础上实现一个集成系统来达到更好的识别能力。

\item 面向三轮对话的情感识别。为了更好地识别一段文本的情感,引入上下文信息被认为有指导性作用。而在文本对话的场景下,要识别某个人在某次发言所表达的情感,可以考虑引入该次聊天的历史记录。本文研究了在两人轮流发言的三轮对话中,识别最后一轮发言所表达的情感。此处采用与面向微博的反讽识别相同的系统设计以评估设计方案的泛用性,另外我们进一步探索了上下文信息的运用方面,透过把不同的运用方面体现在分类器的不同设计方案上,并根据其性能差别和对神经网络权重的分析,了解上下文信息如何有助于识别正文的情感。
\end{itemize}

% 关键词是为了文献标引工作、用以表示全文主要内容信息的单词或术语。关键词不超过 5
% 个,每个关键词中间用分号分隔。(模板作者注:关键词分隔符不用考虑,模板会自动处
% 理。英文关键词同理。)

\end{cabstract}

% 如果习惯关键字跟在摘要文字后面,可以用直接命令来设置,如下:
% \ckeywords{\TeX, \LaTeX, CJK, 模板, 论文}

\begin{eabstract}
   An abstract of a dissertation is a summary and extraction of research work
   and contributions. Included in an abstract should be description of research
   topic and research objective, brief introduction to methodology and research
   process, and summarization of conclusion and contributions of the
   research. An abstract should be characterized by independence and clarity and
   carry identical information with the dissertation. It should be such that the
   general idea and major contributions of the dissertation are conveyed without
   reading the dissertation.

   An abstract should be concise and to the point. It is a misunderstanding to
   make an abstract an outline of the dissertation and words ``the first
   chapter'', ``the second chapter'' and the like should be avoided in the
   abstract.

   Key words are terms used in a dissertation for indexing, reflecting core
   information of the dissertation. An abstract may contain a maximum of 5 key
   words, with semi-colons used in between to separate one another.
\end{eabstract}

% \ekeywords{\TeX, \LaTeX, CJK, template, thesis}
