\thusetup{
  %******************************
  % 注意:
  %   1. 配置里面不要出现空行
  %   2. 不需要的配置信息可以删除
  %******************************
  %
  %=====
  % 秘级
  %=====
  secretlevel={秘密},
  secretyear={10},
  %
  %=========
  % 中文信息
  %=========
  ctitle={面向社交文本的情感识别研究},
  cdegree={工学硕士},
  cdepartment={计算机科学与技术系},
  cmajor={计算机科学与技术},
  cauthor={梁锡豪},
  csupervisor={徐明星副教授},
  % cassosupervisor={陈文光教授}, % 副指导老师
  % ccosupervisor={某某某教授}, % 联合指导老师
  % 日期自动使用当前时间,若需指定按如下方式修改:
  % cdate={超新星纪元},
  %
  % 博士后专有部分
  % cfirstdiscipline={计算机科学与技术},
  % cseconddiscipline={系统结构},
  % postdoctordate={2009年7月——2011年7月},
  % id={编号}, % 可以留空: id={},
  % udc={UDC}, % 可以留空
  % catalognumber={分类号}, % 可以留空
  %
  %=========
  % 英文信息
  %=========
  etitle={Sentiment Analysis in Social Text},
  % 这块比较复杂,需要分情况讨论:
  % 1. 学术型硕士
  %    edegree:必须为Master of Arts或Master of Science(注意大小写)
  %             “哲学、文学、历史学、法学、教育学、艺术学门类,公共管理学科
  %              填写Master of Arts,其它填写Master of Science”
  %    emajor:“获得一级学科授权的学科填写一级学科名称,其它填写二级学科名称”
  % 2. 专业型硕士
  %    edegree:“填写专业学位英文名称全称”
  %    emajor:“工程硕士填写工程领域,其它专业学位不填写此项”
  % 3. 学术型博士
  %    edegree:Doctor of Philosophy(注意大小写)
  %    emajor:“获得一级学科授权的学科填写一级学科名称,其它填写二级学科名称”
  % 4. 专业型博士
  %    edegree:“填写专业学位英文名称全称”
  %    emajor:不填写此项
  edegree={Master of Science},
  emajor={Computer Science and Technology},
  eauthor={Liang Xihao},
  esupervisor={Professor Xu Mingxing},
  % eassosupervisor={Chen Wenguang},
  % 日期自动生成,若需指定按如下方式修改:
  % edate={December, 2005}
  %
  % 关键词用“英文逗号”分割
  ckeywords={情感分析, 反讽识别, 深度学习, 集成学习},
  ekeywords={Sentiment analysis, irony detection, deep learning, ensemble learning}
}

% 定义中英文摘要和关键字
\begin{cabstract}

自Web2.0普及后,人们逐渐习惯在互联网的各种平台上分享他们的想法和情感。透过对这些媒体进行情感分析,可以得知人们对特定人事物的想法和态度。对人们的想法快速作出响应能够带来相应的商业价值和政治价值,其相关技术也因此得到了重视。文本作为社交平台上的主要媒体之一,面向文本的情感识别在近年也成为了热门的研究领域。本论文的主要内容如下:

\begin{enumerate}

\item {\bf 多分类器分层识别算法}。随着现实中应用场景变得复杂,需要解决的多分类问题越来越多。当区分的类别越多,机器学习算法对数据进行拟合的难度越高,另外对识别性能的要求也变得复杂。为此,我们提出了一种多分类器分层识别算法框架,把一个多分类问题拆解成多个子分类问题的叠加,再透过逐步回答每个子问题得出识别结果。此算法框架有以下优点:首先,由于识别过程中每一步只关注一个子分类问题,整个算法的识别性能能从局部进行调整,以满足特定的性能要求;其次,算法中的每个组成部分可以分别采用不同的模型,以此结合不同模型的建模能力;最后,此算法框架与研究的媒体无关,适用于其他领域的多分类问题。本文基于国际比赛SemEval-2018任务三的微博反讽识别进行实验,结果显示算法超过了当时排名第一的系统,而透过对算法中间结果的分析验证了算法设计的合理性。

\item {\bf 结合上下文的多通道模型}。在某些场景下,仅凭一段文本无法准确理解发言者想表达的意思和态度。为了处理这种情况,一些文本情感识别研究会引入上下文作为提示。为此我们提出了一种多通道模型框架,让不同部分的上下文经过不同的编码器来提取特征,再合并正文的信息得出识别结果,以此应付各部分上下文对识别目标起不同作用的情况。我们将此模型应用于面向三轮对话的情感识别,并结合前述的算法框架提出了另一个多分类器分层识别算法。在国际比赛SemEval-2019任务三的参赛结果显示算法达到了当时排名前十的性能,这同时验证了多分类器分层识别算法框架适用于不同的多分类问题。

\end{enumerate}

% 关键词是为了文献标引工作、用以表示全文主要内容信息的单词或术语。关键词不超过 5
% 个,每个关键词中间用分号分隔。(模板作者注:关键词分隔符不用考虑,模板会自动处
% 理。英文关键词同理。)

\end{cabstract}

% 如果习惯关键字跟在摘要文字后面,可以用直接命令来设置,如下:
% \ckeywords{\TeX, \LaTeX, CJK, 模板, 论文}

\begin{eabstract}

Since the popularization of Web 2.0, people get used to sharing their thoughts and emotions on different online platforms. By analyzing the sentiment of these data, the people's opinion towards certain things can be observed. Vast commercial and political values can be obtained through quick response to the people's attitude. Therefore related technologies are getting more attention. As text is one of the mostly used media on social platforms, text sentiment analysis has become a popular research field. This thesis is organized as follows:

\begin{enumerate}

\item {\bf Hierarchical ensemble algorithm framework for multi-classification}. As real-world application scenes are becoming more complex, more multi-classification problems are encountered. As the data are divided into more categories, it becomes more difficult for machine learning algorithms to fit the data. On the other hand, the requirement of predictive performance is getting more complicated. Hence, we propose a hierarchical ensemble algorithm framework for multi-classification, which disassembles a multi-classification problem into a sequence of sub-classification problems and classifies an object by answering the sub-questions step by step. As the algorithm only focuses on a sub-problem at each step, its performance can be adjusted locally to fulfill certain performance requirements. On the other hand, models can be chosen for each sub-problem individually, hence the information captured by different kinds of models can be combined in the algorithm. Furthermore, this algorithm framework is independent of the studied data type, which means it can also be applied to the multi-classification problems in other study fields. Experiments are launched based on SemEval-2018 Task 3, irony detection in English tweets. Results show that our algorithm exceeds the performance of the no. 1 participating system and the analysis of the algorithm's intermediate output proves the rationality of our algorithm design. 

\item {\bf Multi-channel model for manipulating contextual information}. In some case, we can hardly understand what a person wants to express or how he feels only through his written message. To deal with this problem, some of the text sentiment analysis researches take into account the contextual clues. Therefore, we propose a multi-channel model. By feeding contextual information that plays different parts of role into different feature encoders, features of different contextual clues are encoded and further combined with that of the main message to get the identification result. This model is applied to contextual emotion detection in text and another hierarchical ensemble algorithm is proposed based on it. The competition results of SemEval-2019 Task 3 show that our algorithm reaches the top 10 performance among the participants, which also verifies that our algorithm framework can be applied to different multi-classification problems.

\end{enumerate}

\end{eabstract}

% \ekeywords{\TeX, \LaTeX, CJK, template, thesis}
