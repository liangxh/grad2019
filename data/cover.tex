\thusetup{
  %******************************
  % 注意:
  %   1. 配置里面不要出现空行
  %   2. 不需要的配置信息可以删除
  %******************************
  %
  %=====
  % 秘级
  %=====
  secretlevel={秘密},
  secretyear={10},
  %
  %=========
  % 中文信息
  %=========
  ctitle={基于多步决策的文本情感识别},
  cdegree={工学硕士},
  cdepartment={计算机科学与技术系},
  cmajor={计算机科学与技术},
  cauthor={梁锡豪},
  csupervisor={徐明星副教授},
  % cassosupervisor={陈文光教授}, % 副指导老师
  % ccosupervisor={某某某教授}, % 联合指导老师
  % 日期自动使用当前时间,若需指定按如下方式修改:
  % cdate={超新星纪元},
  %
  % 博士后专有部分
  % cfirstdiscipline={计算机科学与技术},
  % cseconddiscipline={系统结构},
  % postdoctordate={2009年7月——2011年7月},
  % id={编号}, % 可以留空: id={},
  % udc={UDC}, % 可以留空
  % catalognumber={分类号}, % 可以留空
  %
  %=========
  % 英文信息
  %=========
  etitle={An Introduction to \LaTeX{} Thesis Template of Tsinghua University v\version},
  % 这块比较复杂,需要分情况讨论:
  % 1. 学术型硕士
  %    edegree:必须为Master of Arts或Master of Science(注意大小写)
  %             “哲学、文学、历史学、法学、教育学、艺术学门类,公共管理学科
  %              填写Master of Arts,其它填写Master of Science”
  %    emajor:“获得一级学科授权的学科填写一级学科名称,其它填写二级学科名称”
  % 2. 专业型硕士
  %    edegree:“填写专业学位英文名称全称”
  %    emajor:“工程硕士填写工程领域,其它专业学位不填写此项”
  % 3. 学术型博士
  %    edegree:Doctor of Philosophy(注意大小写)
  %    emajor:“获得一级学科授权的学科填写一级学科名称,其它填写二级学科名称”
  % 4. 专业型博士
  %    edegree:“填写专业学位英文名称全称”
  %    emajor:不填写此项
  edegree={Doctor of Engineering},
  emajor={Computer Science and Technology},
  eauthor={Liang Xihao},
  esupervisor={Professor Xu Mingxing},
  % eassosupervisor={Chen Wenguang},
  % 日期自动生成,若需指定按如下方式修改:
  % edate={December, 2005}
  %
  % 关键词用“英文逗号”分割
  ckeywords={情感分析, 反讽识别, 深度学习,集成学习},
  ekeywords={Sentiment analysis, irony detection, deep learning, ensemble learning}
}

% 定义中英文摘要和关键字
\begin{cabstract}

自Web2.0普及后,人们习惯在互联网的各种平台上透过不同的媒体来分享和传递他们的想法和情感。透过对这些媒体进行情感分析,我们可以得知人们对特定人事物的想法和态度。譬如透过分析产品评论了解用户对新产品是否满意,又或者分析社交平台上的舆论了解网民对新政策是否同意,以此快速响应,其中蕴含着丰富的商业价值和政治价值,其相关技术的研究价值在近年也得到了重视。而文本作为较常用于发表意见的媒体之一,加之部分微博等社交平台推出了用户生产数据的采集服务,提供了充裕的文本数据资源,面向文本的情感识别研究得以更快的速度发展。本文针对面向文本的情感识别主要作出了以下贡献:

\begin{enumerate}

\item 基于多步决策的多分类系统设计方法。随着现实中应用场景变得复杂,需要解决的多分类问题越来越大,和二分类问题相比,随着需要区分的类别越多,对机器学习算法的拟合能力要求更高,另外对系统的性能要求也变得复杂,譬如确保个别类别的召回率和正确率等。为此,我们提出了一种基于多步决策的多分类系统设计方法,透过把一个多分类问题拆解成多个子分类问题,再透过逐步回答每个子问题得出最终的识别结果。一方面,由于每一步只关注一个子分类问题,所以能局部提高系统的识别能力。另一方面,透过把原问题拆解成了子问题,我们可以调整个别类别之间的区分能力,从而针对特殊的评价指标进行优化。

\item 用于结合上下文的多通道模型。在文本情感识别中,有时候仅凭一段文本无法准确了解发言者想表达的意思和态度,在一些场景下我们会考虑引入文本的上下文作为提示。为此我们提出了一种多通道模型框架,让对识别目标起不同作用的上下文先分别经过不同的编码器进行编码提取出有用的特征,再考虑合并本文和上下文的信息得出识别结果。

\end{enumerate}

为了验证基于多步决策多分类系统框架,我们将其应用于面向微博的反讽识别,根据公开比赛SemEval-2018的任务三进行实验,结果显示我们的模型超过了当时排名第一的系统。
另外为了验证我们提出的多通道分类模型,我们将其应用于面向三轮对话的情感识别,根据公开比赛SemEval-2019的任务三进行实验,结果显示我们的模型达到了当时排名前十的性能。

% 关键词是为了文献标引工作、用以表示全文主要内容信息的单词或术语。关键词不超过 5
% 个,每个关键词中间用分号分隔。(模板作者注:关键词分隔符不用考虑,模板会自动处
% 理。英文关键词同理。)

\end{cabstract}

% 如果习惯关键字跟在摘要文字后面,可以用直接命令来设置,如下:
% \ckeywords{\TeX, \LaTeX, CJK, 模板, 论文}

\begin{eabstract}
   
   pass

\end{eabstract}

% \ekeywords{\TeX, \LaTeX, CJK, template, thesis}
