\begin{resume}

  \resumeitem{个人简历}

  1994 年 6 月 4 日出生于 澳门特别行政区。

  2012 年 9 月考入清华大学计算机科学与技术系,2016 年 7 月本科毕业并获得工学学士学位。

  2016 年 9 月免试进入清华大学计算机科学与技术系攻读硕士学位至今。

  \researchitem{发表的学术论文} % 发表的和录用的合在一起

  % 1. 已经刊载的学术论文(本人是第一作者,或者导师为第一作者本人是第二作者)
  % \begin{publications}
    % \item Yang Y, Ren T L, Zhang L T, et al. Miniature microphone with silicon-based ferroelectric thin films. Integrated Ferroelectrics, 2003, 52:229-235. (SCI 收录, 检索号:758FZ.)
    % \item 杨轶, 张宁欣, 任天令, 等. 硅基铁电微声学器件中薄膜残余应力的研究. 中国机械工程, 2005, 16(14):1289-1291. (EI 收录, 检索号:0534931 2907.)
    % \item 杨轶, 张宁欣, 任天令, 等. 集成铁电器件中的关键工艺研究. 仪器仪表学报, 2003, 24(S4):192-193. (EI 源刊.)
  % \end{publications}

  % 2. 尚未刊载,但已经接到正式录用函的学术论文(本人为第一作者,或者
  %    导师为第一作者本人是第二作者)。
  \begin{publications}[before=\publicationskip,after=\publicationskip]
    % \item Yang Y, Ren T L, Zhu Y P, et al. PMUTs for handwriting recognition. In press. (已被 Integrated Ferroelectrics 录用. SCI 源刊.)

    \item Xihao Liang, Ye Ma, Mingxing Xu. THU-HCSI at SemEval-2019 Task 3: Hierarchical Ensemble Classification of Contextual Emotion in Conversation [C]//Proceedings of The 13th International Workshop on Semantic Evaluation (SemEval-2019). [S.l.: s.n.], 2019: 345-349.

    \item Ye Ma, Xihao Liang, Mingxing Xu. THUHCSI in MediaEval 2018 Emotional Impact of Movies Task[C]//Working Notes Proceedings of the MediaEval 2018 Workshop, Sophia Antipolis, France, 29-31 October 2018. [S.l.: s.n.], 2018.

  \end{publications}

  % 3. 其他学术论文。可列出除上述两种情况以外的其他学术论文,但必须是
  %    已经刊载或者收到正式录用函的论文。
  % \begin{publications}
    % \item Wu X M, Yang Y, Cai J, et al. Measurements of ferroelectric MEMS microphones. Integrated Ferroelectrics, 2005, 69:417-429. (SCI 收录, 检索号:896KM)
    % \item 贾泽, 杨轶, 陈兢, 等. 用于压电和电容微麦克风的体硅腐蚀相关研究. 压电与声光, 2006, 28(1):117-119. (EI 收录, 检索号:06129773469)
    % \item 伍晓明, 杨轶, 张宁欣, 等. 基于MEMS技术的集成铁电硅微麦克风. 中国集成电路, 2003, 53:59-61.
  % \end{publications}

  % \researchitem{研究成果} % 有就写,没有就删除
  % \begin{achievements}
  %   \item 任天令, 杨轶, 朱一平, 等. 硅基铁电微声学传感器畴极化区域控制和电极连接的
  %    方法: 中国, CN1602118A. (中国专利公开号)
  %  \item Ren T L, Yang Y, Zhu Y P, et al. Piezoelectric micro acoustic sensor
  %    based on ferroelectric materials: USA, No.11/215, 102. (美国发明专利申请号)
  %\end{achievements}

  \researchitem{参加的科研项目} % 发表的和录用的合在一起

  \begin{achievements}

  \item 国家自然科学基金重点项目:互联网话语理解的心理机制与计算建模(资助号:61433018)

  \end{achievements}


\end{resume}
