\chapter{总结和展望}
\label{cha:conclusion}

\section{论文工作总结}

随着互联网用户越来越习惯于在社交媒体上发言,对这些数据进行内容分析的价值逐渐受到重视,相关研究得以蓬勃发展,面向文本的情感识别研究就是其中之一。本硕士论文探索了该领域下的两个核心问题:数据不均匀在多分类问题中的影响、在算法建模中引入上下文信息。为了解决以上两个问题,我们分别提出了一种基于多步决策的多分类系统设计方法以及一种多通道分类模型。本论文的主要研究工作如下:

\begin{enumerate}

\item {\bf 基于多步决策的微博反讽识别}。我们首先对和情感识别紧密相关的反讽识别进行研究,并指出在真实场景中数据不均匀的问题。



\item {\bf 面向微博的反讽识别}。对文本的修辞手法识别,我们选择了在意图识利中具有代表性的反讽修辞手法作为识别对目标,识别反讽修辞的使用对于避免错误了解发言者的意图有重要的作用。为此,我们基于国际比赛SemEval-2018的任务三进行实验。对于其中的子任务一,即面向微博文本是否带有反讽的二分类问题,我们的系统达到了当时参赛系统中靠前的性能,并和我们当时参赛的系统对比有了显着的进步。对于其中的子任务二,即把反讽修辞细分成三个类别后的四分类问题,我们提出了一个多步决策的反讽识别系统,透过针对原问题中的多个子分类问题进行建模,逐步修正对应类别样本的识别结果。实验结果显示我们系统在主要评价指标F1值上显着优于当时参赛系统中的第一,而在其他几项指标上都达到了仅次于第一名的数值,验证了我们提出的多步决策识别系统的有效性。另外基于系统中的每步中间结果在测试集上的表现,我们分析了多步决策对整体识别性能带多提升的原因,进一步解释了我们系统框架设计的合理性。

\item {\bf 面向三轮对话的情感识别}。在一些场景下,单凭发言者说的内容有可能无法完全理解他想表达的想法或情感,为此引入上下文就成为突破点之一。在假设上下文和发言者当前的发言有关的前提下,如何结合上下文信息用于意图识别就成为了一个备受关注的课题。为此,我们选择了面向三轮对话的情感识别作为研究的切入点,基于国际比赛SemEval-2019的任务三进行实验。首先经过人工观察,我们分析了三轮对话中每轮发言对识别最后一轮的情感是否有帮助、它们如何起到提示的作用、以及它们的作用是否相同。基于分析结果,我们提出了一个三通道的模型框架对三轮对话进行建模。针对该任务对应的四分类情感识别问题,我们基于和研究面向微博的反讽识别中相似的方法提出了一个多步决策情感识别系统,同样地每一步只关注原问题中的一个子问题,对识别结果逐步进行修正。参赛结果显示我们的系统达到所有165个参赛系统中排名前十的性能。基于对子分类问题上各模型的性能比较,我们指出对于不同类别的样本,其背后的数学模型可能有所不同,这也是我们系统中对不同子分类问题采用不同模型来得出分类器的理由。另外,我们也观察了系统在测试集上经过对识别结果的每一步修正后整体的性能变化,从而解释了我们在设计多步决策系统时选择子分类问题的原则,并强调了子分类问题的选择和数据集中各类别样本量的分布以及评价指标紧密相关。

\end{enumerate}

\section{未来工作展望}

面向文本的情感识别技术在实际应用中有明显的不足之处,相关研究依然存在很大的进步空间。特别是随着信息技术发展,文本的使用场景只会变得越来越复杂。总结第\ref{cha:exp_irony_det}章和第\ref{cha:exp_context_emo}章我们对系统的错误分析,我们为面向文本情感识别的后续工作给出以下建议:

\begin{enumerate}

\item {\bf 多模型融合}。在两个章节的实验中,我们都指出了基于现阶段主流的人工神经网络可能无法同时对多个类别的数据进行拟合,原因在于各个类别背后的语言特征有本质上的不同。尝试提出拟合能力更强的人工神经网络结构固然是前进方向之一,但我们同时鼓励采用较简单的模型对局部数据进行建模,再透过类似我们提出的多步决策框架来结合各个模型捕足到的信息,以此解决单个模型建模能力有限的问题。
。

\item {\bf 引入语义知识}。文本情感识别的核心难点之一在于语义。机器学习方法只能从训练数据中尝试学习词(组)的语义,但应用场景中往往会出现训练数据中没有出现过的词(组),导致算法无法理解其中的信息。为了处理这部分内容,我们建议引入更多额外的语义知识,以建立在训练数据中出现过和没有出现过的词(组)之间的联系。词嵌入方法虽然在原理上能关联语义相近的词,但从实验结果来看依然有不足之处(如无法处理上一词多义的情况),这些问题都有待进一步探索。

\item {\bf 引入语用知识}。文本情感识别的核心难点其二在于丰富多样的语用,譬如反讽和比喻等修辞手法。如果没有意识到发言者使用了特殊的修辞手法,单凭字面意思了解文本内容就有可能忽略甚至曲解发言者本身想表达的意思。而为了识别修辞的使用,本文中尝试采用多种人工神经网络模型进行建模,但在对系统进行错误分析时,我们会发现一些显而易见的模式未能被系统正确识别。对于这部分样本,最直观的做法是采用基于规则的方法。修辞手法的具体使用虽然多种多样,但根据语用知识我们可以总结出其中有限种模式。进一步地,有别于单纯基于规则的方法,我们建议结合基于规则的方法和机器学习方法,优先召回能被规则匹配的样本,再由机器学习方法识别难以由人工归纳出规则的样本,以此减轻人工设计规则的压力。

\end{enumerate}







