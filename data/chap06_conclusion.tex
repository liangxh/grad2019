\chapter{总结和展望}
\label{cha:conclusion}

\section{论文工作总结}

随着互联网用户越来越习惯于在社交媒体上发言,带来了大量的研究数据,加上商业上的应用需求增加,相关研究得以蓬勃发展,而面向文本的意图识别则是其中备受关注的研究领域之一。本硕士论文探索了该领域下的两个核心问题:对文本的修辞手法识别、结合上下文进行意图识别。对这两个问题,我们分别选择反讽识别和多轮对话中的情感识别作为研究的切入点,并分别以对应研究问题的国际比赛的配置和数据集进行实验。一方面对问题本身进行深入分析,另一方面透过和其他参赛系统的性能进行比较来评估我们系统的性能水平。

\begin{enumerate}

\item {\bf 面向微博的反讽识别}。对文本的修辞手法识别,我们选择了在意图识利中具有代表性的反讽修辞手法作为识别对目标,识别反讽修辞的使用对于避免错误了解发言者的意图有重要的作用。为此,我们基于国际比赛SemEval-2018的任务三进行实验。对于其中的子任务一,即面向微博文本是否带有反讽的二分类问题,我们的系统达到了当时参赛系统中靠前的性能,并和我们当时参赛的系统对比有了显着的进步。对于其中的子任务二,即把反讽修辞细分成三个类别后的四分类问题,我们提出了一个多步决策的反讽识别系统,透过针对原问题中的多个子分类问题进行建模,逐步修正对应类别样本的识别结果。实验结果显示我们系统在主要评价指标F1值上显着优于当时参赛系统中的第一,而在其他几项指标上都达到了仅次于第一名的数值,验证了我们提出的多步决策识别系统的有效性。另外基于系统中的每步中间结果在测试集上的表现,我们分析了多步决策对整体识别性能带多提升的原因,进一步解释了我们系统框架设计的合理性。

\item {\bf 面向三轮对话的情感识别}。在一些场景下,单凭发言者说的内容有可能无法完全理解他想表达的想法或情感,为此引入上下文就成为突破点之一。在假设上下文和发言者当前的发言有关的前提下,如何结合上下文信息用于意图识别就成为了一个备受关注的课题。为此,我们选择了面向三轮对话的情感识别作为研究的切入点,基于国际比赛SemEval-2019的任务三进行实验。首先经过人工观察,我们分析了三轮对话中每轮发言对识别最后一轮的情感是否有帮助、它们如何起到提示的作用、以及它们的作用是否相同。基于分析结果,我们提出了一个三通道的模型框架对三轮对话进行建模。针对该任务对应的四分类情感识别问题,我们基于和研究面向微博的反讽识别中相似的方法提出了一个多步决策情感识别系统,同样地每一步只关注原问题中的一个子问题,对识别结果逐步进行修正。参赛结果显示我们的系统达到所有165个参赛系统中排名前十的性能。基于对子分类问题上各模型的性能比较,我们指出对于不同类别的样本,其背后的数学模型可能有所不同,这也是我们系统中对不同子分类问题采用不同模型来得出分类器的理由。另外,我们也观察了系统在测试集上经过对识别结果的每一步修正后整体的性能变化,从而解释了我们在设计多步决策系统时选择子分类问题的原则,并强调了子分类问题的选择和数据集中各类别样本量的分布以及评价指标紧密相关。

\end{enumerate}

\section{未来工作展望}

虽然面向文本的意图识别已经经功了数十年的研究,但依然有很大的进步空间,特别是随着信息技术发展,文本的使用场景变得更多更复杂,研究工作还会不断继续。在第\ref{cha:exp_irony_det}章和第\ref{cha:exp_context_emo}章,我们分别研究了对面向微博的反讽识别以及面向三轮对话的情感识别,
其中除了对不同模型的性能进行了评估,还对系统的错误进行了分析,以下我们将其中的结论,为面向文本的意图识别研究给出后续工作的建议:

\begin{enumerate}

\item {\bf 多模型融合}。在两个章节的实验中我们都指出了基于现阶段主流的人工神经网络无法同时对所有类别的数据进行拟合。尝试提出拟合能力更强的人工神经网络结构固然是前进方向之一,但我们同时鼓励采用较简单的模型对局部数据进行建模和识别,再透过多步決策得出全局的识別结果,以此降低对模型拟合能力的要求,但另一方面则要求研究人员了解如何拆解一个多分类问题,从中选出最有利于核心评价指标的拆解方案。

\item {\bf 引入语言知识}。面向文本的意图理解的核心难点之一在于丰富多样的语言用法,譬如反讽和比喻等修辞手法。如果没有意识到发言者使用了修辞手法,单凭字面意思了解文本内容就有可能忽略甚至曲解发言者本身想表达的意思。而为了识别修辞的使用,本文中尝试采用各种现有模型来建模,但在系统错误分析时会发现部分模式的样本被系统所误判。对于这部分样本,最直观的做法是采用基于规则的方法。修辞手法的具体使用虽然多种多样,但根据语言知识我们可以总结出其中有限种模式。进一步地,有别于单纯基于规则的方法,我们建议结合基于规则的方法和机器学习方法,优先召回能被规则匹配的样本,再由机器学习方法区分难以由人工归纳出明显规则的样本,以此减轻人工设计规则的压力。

\item {\bf }。 

\end{enumerate}







