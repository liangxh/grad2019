\chapter{总结和展望}
\label{cha:conclusion}

\section{论文工作总结}

随着互联网用户越来越习惯于在社交媒体上发言,对这些数据进行内容分析的价值逐渐受到重视,相关研究得以蓬勃发展,面向文本的情感识别研究就是其中之一。本硕士论文探索了该领域下的两个核心问题:多分类问题中的数据不均匀、在算法建模中引入上下文信息。为了解决以上两个问题,我们分别提出了一种多分类器分层识别算法框架以及一种多通道分类模型。本论文的主要研究工作如下:

\begin{enumerate}

\item {\bf 基于多分类器分层的微博反讽识别}。我们首先对和情感识别紧密相关的反讽识别进行研究,并指出在真实场景中数据不均匀的问题,为此我们提出了一种多分类器分层识别算法,把原本的多分类问题拆解成多个子分类问题的叠加,再透过逐步回答每个子分类问题得出最终的反讽识别结果。我们基于国际比赛SemEval-2018任务三的两个子任务进行实验。对于其中的子任务一,即识别微博文本是否带有反讽的二分类问题,我们训练了一组以2层BiLSTM作为模型的二分类器,再透过多数投票得出识别结果。实验结果显示我们的系统达到了当时参赛系统中靠前的性能,并和我们当时参赛的系统对比有了明显的进步。对子任务二,即在子任务一的基础上把反讽细分成三个类别后的四分类问题,我们采用了我们提出的多分类器分层识别算法。实验结果显示我们的系统在主要评价指标F1值上显着优于当时参赛系统中的第一,而在其他几项指标上都达到了仅次于第一名的数值,验证了我们系统的性能。另外透过对系统中间结果进行观察,我们解释了算法中每一步如何对识别系统的整体性能带来贡献,同时验证了我们算法设计的合理性。

\item {\bf 基于多通道模型引入上下文的情感识别}。为了研究如何在情感识别中引入上下文,我们研究了面向三轮对话的情感识别,同时提出了一种多通道分类模型,以结合起不同作用的上下文。我们基于国际比赛SemEval-2019的任务三进行实验。首先透过对语料进行观察,我们分析了三轮对话中每轮发言对识别最后一轮情感的作用,以此提出了一个三通道模型,再结合前面提出的算法设计框架,我们给出了一个面向情感识别的多分类器分层识别算法。参赛结果显示我们的系统在165个参赛系统中排名前十。另外透过对系统的中间结果进行分析,我们验证了算法中每一步的作用,同时证明了我们的多分类器分层识别算法框架通用于不同的多分类问题,但需要针对核心评价指标和各个类别的样本分布作具体设计。

\end{enumerate}

\section{未来工作展望}

面向文本的情感识别技术在实际应用中有明显的不足之处,相关研究依然存在很大的进步空间。特别是随着信息技术发展,文本的使用场景只会变得越来越复杂。总结第\ref{cha:exp_irony_det}章和第\ref{cha:exp_context_emo}章我们对系统的错误分析,我们为面向文本情感识别的后续工作给出以下建议:

\begin{enumerate}

\item {\bf 多模型融合}。在两个章节的实验中,我们都指出了基于现阶段主流的人工神经网络可能无法同时对多个类别的数据进行拟合,原因在于各个类别背后的语言特征有本质上的不同。尝试提出拟合能力更强的人工神经网络结构固然是前进方向之一,但我们同时鼓励采用较简单的模型对局部数据进行建模,再透过多分类器分层识別的方式结合各个模型捕足到的信息,以此解决单个模型建模能力有限的问题。

\item {\bf 引入语义知识}。文本情感识别的核心难点之一在于语义。机器学习方法只能从训练数据中尝试学习词(组)的语义,但应用场景中往往会出现训练数据中没有出现过的词(组),导致算法无法理解其中的信息。为了处理这部分内容,我们建议引入更多额外的语义知识,以建立在训练数据中出现过和没有出现过的词(组)之间的联系。词嵌入方法虽然在原理上能关联语义相近的词,但从实验结果来看依然有不足之处(如无法处理一词多义的情况),这些问题都有待进一步探索。

\item {\bf 引入语用知识}。文本情感识别的核心难点其二在于丰富多样的语用,譬如反讽和比喻等修辞手法。如果没有意识到发言者使用了特殊的修辞手法,单凭字面意思了解文本内容就有可能忽略甚至曲解发言者本身想表达的意思。而为了识别修辞的使用,本文中尝试采用多种人工神经网络模型进行建模,但在对系统进行错误分析时,我们会发现一些显而易见的模式未能被系统正确识别。对于这部分样本,最直观的做法是采用基于规则的方法。修辞手法的具体使用虽然多种多样,但根据语用知识我们可以总结出其中有限种模式。进一步地,有别于单纯基于规则的方法,我们建议结合基于规则的方法和机器学习方法,优先召回能被规则匹配的样本,再由机器学习方法识别难以由人工归纳出规则的样本,以此减轻人工设计规则的压力。

\end{enumerate}







