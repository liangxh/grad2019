\chapter{面向微博的反讽识别}
\label{cha:exp_irony_det}

\section{本章引论}

社交媒体的发展对我们的语言体系带来了很大的影响,网络上出现了很多新颖的用词和句式,语言的表达方式越来越丰富,也越来越复杂。而反讽是在网络上常见的语言修饰手法之一, 这为反讽相关的研究带来了充足的数据基础。Henry Watson Fowler在《The King's English》一书中指出反讽的使用使得“表面意思和实际意思不同”。譬如一个人说“你这想法真有创意”,在字面意思上是对另一个人的赞同,但在特定背景下,如后接一句“你真相信这能实现吗”,那么发言者实际上可能暗示这个想法无法落地,表面上称赞为“有创意”,其实是指责这种想法不切实际。这在意图识别当中尤其重要,忽略反讽的使用会导致对内容的错误理解,而且这种理解是和真实意思截然相反的,因此识别出反讽的使用或许对相关的场景如情感识别、人机交互能起着正面的作用。

根据Joshi等人\cite{joshi2017automatic}对近年相关研究的总结,反讽识别可以大致分成基于规则的方法和基于机器学习的方法。基于规则的方法透过人工找出反讽中的语言规律,设计出对应的模式,然后在新样本中尝试识别出相应的模式出现。和机器学习方法对比,基于规则的方法优点在于无需模型训练,但要求研究员对反讽有充分的语言理解,设计的模式对样本的复盖程度决定了算法的识别能力。而随着近年深度学习快速发展,一些研究更专注于对词嵌入向量的使用以及人工神经网络的设计和选择。

国际比赛SemEval-2018的任务三\cite{van2018semeval}旨在促进英语微博中的反讽识别研究,其中包含了两个子任务。子任务一是二分类的反讽识别,需要识别微博是否有使用反讽。子任务二是四分类的反讽识别,是子任务一的拓展,除了判断微博是否包含反讽,反讽再细分成三个类别:基于相反语义的言语反讽 
、其他言语反讽、情景反讽。本章节中我们将基于SemEval-2018的任务三进行实验,采用比赛组织者提供的训练数据和测试数据,并透过和其他参赛系统进行比较来评估我们提出的框架的性能。

本章的内容安排如下。在章节\ref{sec:exp_irony_det_data}中我们首先对实验数据进行观察,分析微博文本的特性以及各个反讽类别之间的不同。在章节\ref{sec:exp_irony_det_format}中,我们会基于章节\ref{sec:global_problem_analysis}给出当前问题的形式化表示。在章节\ref{sec:exp_irony_det_framework}中,我们会基于章节\ref{sec:global_framework}的框架给出我们对当前问题的系统框架。最后在章节\ref{sec:exp_irony_det_exp}给出实验的细节,以及对实验结果进行分析。

\section{数据观察}
\label{sec:exp_irony_det_data}

pass

\section{形式化表示}
\label{sec:exp_irony_det_format}

pass

\section{框架设计}
\label{sec:exp_irony_det_framework}

pass

\section{实验与分析}
\label{sec:exp_irony_det_exp}

pass

\subsection{数据预处理}

pass

\subsection{实验设置}

pass

\subsection{模型训练}

pass

\subsection{结果与分析}

pass

% \subsection{错误分析}

\section{本章小结}

pass

